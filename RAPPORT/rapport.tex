% !TEX encoding = IsoLatin

\documentclass{report}
\usepackage[usenames,dvipsnames]{color}
%pour le code
\usepackage{listings}

\usepackage[utf8]{inputenc}
 \usepackage{lmodern}
\usepackage{listings}
\lstset{language=C}
 
 %pour les couleurs !!
 \usepackage[usenames,dvipsnames]{color}

\usepackage{geometry}
\geometry{scale=0.8, nohead}
\usepackage{t1enc}
\usepackage[french]{babel}

\usepackage{url}


 %pour les images
\usepackage{graphicx}
\definecolor{vert}{rgb}{0.2,0.6,0.4} 
\lstset{language=c, breaklines=true, basicstyle=\ttfamily,basicstyle=\scriptsize, keywordstyle=\color{blue}, commentstyle=\color{vert}, stringstyle=\color{red}, identifierstyle=\ttfamily}


\makeatletter

\geometry{ hmargin=2cm, vmargin=2cm}


\makeatletter

\geometry{ hmargin=2cm, vmargin=2cm}

\def\blurb{%
BE Validation de modèles de procédé}


\def\ligne#1{%
  \hbox to \hsize{%
    \vbox{\centering #1}}}%

\makeatletter
\def\maketitle{%
	\null
	\vfill
	\vbox{\centering \Large \textbf{\blurb}}
	\vspace{15mm}
	\vbox{\centering \LARGE \textbf{\@title}}
	\vspace{15mm}
	\vbox{\centering \@author}
	\vspace{8mm}
	\vbox{\centering \@date}
	\vspace{15mm}
	\vbox{\centering \textbf{Résumé du projet}}
	\vspace{5mm}
	\vbox{%
		\setlength{\fboxsep}{10pt}
		\centering \fbox{%
		\begin{minipage}{0.9\textwidth}
			\setlength{\parindent}{1cm}
			\setlength{\parskip}{2ex plus .4ex minus .4ex}                        
			L’objectif de ce mini-projet est de compléter le travail fait en BE IDM et de produire une
                        chaîne de vérification de modèles de processus SimplePDL en utilisant les outils de modelchecking
                        disponibles dans la boîte à outils Tina.
		\end{minipage}%
	}
%
	}
	\vfill
	\clearpage
}

\title{Rapport BE IDM 2012}
\author{
Alexandre Escudero \texttt{<escudero@etud.insa-toulouse.fr>}\\
Mathieu Othacéhé \texttt{<othacehe@etud.insa-toulouse.fr>}}
\date{5ème année SEC}

\begin{document}

\maketitle
\tableofcontents


\nocite{*}

\chapter{Introduction}

Dans ce BE, il s'agit d'utiliser les outils de méta-modélisation et de vérification des modèles afin de modéliser des processus déclinés en activités dépendant de ressources et évoluant dans le temps. Différents outils mis à notre disposition nous permettent réaliser ces opérations qui sont décrites dans ce rapport :
\begin{itemize}
\item Eclipse
\begin{itemize}
\item Ecore
\item ATL
\item Xtext
\item GMF
\item Epsilon
\end{itemize}
\item Tina
\begin{itemize}
\item Selt
\end{itemize}
\end{itemize}

\chapter{Partie 1 : Intégration du temps et des ressources}

\section{T1 : Définition du méta-modèle}

Il s'agit dans cette première section de prendre en compte les ressources et le temps dans le méta-modèle SimplePDL.
On reprend donc le fichier SimplePDL.ecore définissant le méta-modèle, et l'on y ajoute de nouveaux éléments.

\subsection{Ajout des ressources}

Pour la gestion des ressources :\\

On rajoute deux nouvelles classes : \textit{RessourceDefinition} et \textit{RessourceInstance}.
La classe \textit{RessourceDefinition} sert à définir les différentes ressources et leur nombre total disponible dans le système. Par exemple, si l'on dispose de 4 ressources de type machine,
on va alors déclarer dans le fichier définissant le méta-modèle SimplePDL une \textit{RessourceDefinition}, avec un attribut number égal à 4.\\

La classe \textit{RessourceInstance} permet d'exprimer le fait qu'une activité va devoir utiliser une certaine quantité de ressources pour fonctionner.
Elle possède un lien vers une \textit{RessourceDefinition} exprimant le type de la ressource ainsi qu'un lien vers l'activité (\textit{WorkDefinition}) en question.
Elle doit aussi posséder un attribut entier indiquant le nombre de ressources nécéssaires à l'activité.\\

Si l'on affiche le méta-modèle sous forme textuelle (améliorée), on a donc :\\

\begin{figure}[!h] 
\begin{center}
\includegraphics[width=10cm]{Capture-2.png}
\caption{Fichier Ecore modifié} 
\label{img1} 
\end{center}
\end{figure} 

On peut également afficher le méta-modèle sous forme graphique Ecorediag :\\

\begin{figure}[!h] 
\begin{center}
\includegraphics[width=15cm]{Capture-4.png}
\caption{Diagramme Ecore modifié} 
\label{img1} 
\end{center}
\end{figure} 

On se retrouve dans une situation assez symétrique avec les classes \textit{WorkDefinition} et \textit{WorkSequence}. En effet, ces deux classes héritent toutes deux de \textit{ProcessElement} et une \textit{WorkSequence} constitue un lien entre deux \textit{WorkDefinition}.\\

Pour la modélisation des ressources disponibles et de leur utilisation par des activités, les classes \textit{RessourceDefinition} et \textit{RessourceInstance} héritent également de \textit{ProcessElement} et une \textit{RessourceInstance} représentera le lien entre une \textit{RessourceDefinition} et une \textit{WorkDefinition}.

\subsection{Ajout du temps}

L'ajout du temps est plus aisé (à ce niveau là en tout cas). Il suffit d'ajouter des attributs \textit{temps\_min} et \textit{temps\_max} aussi bien au niveau du \textit{Process} que de la \textit{WorkDefinition}. Le fichier Ecore modifié est le suivant :\\

\begin{figure}[!h] 
\begin{center}
\includegraphics[width=10cm]{Capture-5.png}
\caption{Ecore avec gestion du temps} 
\label{img1} 
\end{center}
\end{figure} 

A présent que le fichier Ecore est mis à jour par rapport aux nouvelles spécifications, on va pouvoir modifier l'éditeur graphique GMF pour tenir compte des nouveautés ajoutées.

\newpage

\section{T2 : Syntaxe concrète graphique}

Il s'agit ici d'adapter l’éditeur graphique SimplePDL créé en TP pour permettre de saisir graphiquement les ressources disponibles et l’allocation des ressources nécessaires à chaque activité. On doit également pouvoir spécifier les propriétés temporelles de chaque activité et du processus global.

Pour mettre en oeuvre cet outil graphique on utilise le plugin GMF d'Eclipse qui permet de définir les différentes composantes de l'éditeur à partir du méta-modèle Ecore défini précédemment que l'on considère ici comme une syntaxe abstraite. Après avoir paramétré les composants, l'éditeur graphique permet de construire des processus en définissant des activités, des relations entre elles, des ressources et leurs relations avec les activités. On peut donc reconstruire le modèle suivant, correspondant à l'exemple décrit dans le sujet :

\begin{figure}[!h] 
\begin{center}
\includegraphics[width=15cm]{Capture-6.png}
\caption{Modèle défini dans le sujet} 
\label{img1} 
\end{center}
\end{figure} 

\begin{figure}[!h] 
\begin{center}
\includegraphics[width=10cm]{Capture-8.png}
\caption{Exemple d'utilisation de l'éditeur graphique engendré} 
\label{img1} 
\end{center}
\end{figure}

On peut observer dans ce modèle édité graphiquement, que l'on représente :
\begin{itemize}
\item un élément de type \textit{WorkDefinition} par une ellipse
\item un élément de type \textit{WorkSequence} par une flèche entre deux éléments de type \textit{WorkDefinition}
\item un élément de type \textit{RessourceDefinition} par un rectangle
\item un élément de type \textit{RessourceInstance} par une flèche entre un élément de type \textit{RessourceDfinition} et un de type \textit{WorkDefinition}\\
\end{itemize}

Les propriétés telles que :
\begin{itemize}
\item le nombre de ressources des éléments de type \textit{RessourceDefinition}
\item le nombre des éléménets de type \textit{RessourceInstance}
\item le type des relations de type \textit{WorkSequence}
\item les intervalles de temps des éléments de type \textit{WorkDefinition} et \textit{Process}
\end{itemize}
sont accessibles depuis l'onglet Propriétés.

\begin{figure}[!h] 
\begin{center}
\includegraphics[width=10cm]{Capture-10.png}
\caption{Propriétés d'une RessourceInstance} 
\label{img1} 
\end{center}
\end{figure} 

Le modèle généré graphiquement peut être également représenté sous forme arborescente pour être éventuellement exporté ensuite vers un format générique tel que XMI :

\begin{figure}[!h] 
\begin{center}
\includegraphics[width=10cm]{Capture-9.png}
\caption{Modèle SimplePDL sous forme arborescente généré par l'éditeur graphique} 
\label{img1} 
\end{center}
\end{figure} 

\newpage

\section{T3 : Syntaxe concrète textuelle}

Nous devons donner une syntaxe concrète textuelle de SimplePDL pour prendre en compte les ressources et le temps et l’outiller avec xText. On va donc reprendre la grammaire xText élaborée lors des séances précédentes.

On rajoute ces lignes dans le fichier xText pour prendre en compte dans la syntaxe les éléments de type \textit{RessourceDefinition} et \textit{RessourceInstance} :\\

\begin{verbatim}
RessourceDefinition returns RessourceDefinition:
        'rd'
                name=EString
                'number' number=EInt
        ;

RessourceInstance returns RessourceInstance:
        'ri'
                'type' type=[RessourceDefinition|EString]
                'activity' activity=[WorkDefinition|EString]
                'instances' instances=EInt
        ;
\end{verbatim}

On définit ainsi qu'une RessourceDefinition possède un nom et un chiffre indiquant sa quantité. Un élément de type \textit{RessourceInstance} possède un type, qui n'est autre qu'un élément de type \textit{RessourceDefinition}.
Il possède aussi une activité qui correspond à un élément de type \textit{WorkDefinition} et un nombre d'instances entier.\\

Il faut aussi modifier la ligne \textit{ProcessElement} pour indiquer que l'on peut déclarer des éléments de type \textit{RessourceDefinition} et des \textit{RessourceInstance} à l'intérieur de l'élément racine de type \textit{Process}.

\begin{verbatim}
ProcessElement returns ProcessElement:
        WorkDefinition | WorkSequence | Guidance | RessourceDefinition | RessourceInstance
        ;
\end{verbatim}

Enfin, on ajoute au \textit{Process} et à la \textit{WorkDefinition} des attributs \textit{min\_time} et \textit{max\_time}.

\begin{verbatim}
WorkDefinition returns WorkDefinition:
        'wd'
                name=EString
                'min' min_time=EInt
                'max' max_time=EInt
        ;
\end{verbatim}

et

\begin{verbatim}
Process returns Process:
        {Process}
        'process'
                name=EString
                '{'
                        'min_time' min_time=EInt
                        'max_time' max_time=EInt
                        (processElements+=ProcessElement (processElements+=ProcessElement)* )?
                '}'
        ;
\end{verbatim}

On peut à présent créer un fichier s'appuyant sur la grammaire créée et définir un modèle de processus, notamment celui de l'exemple de la partie précédente :

\begin{verbatim}
process exemple {
        min_time 20
        max_time 50
        wd Conception min 10 max 16
        wd RedactionDoc min 8 max 12
        wd Development min 12 max 14
        wd RedactionTest min 10 max 12
        ws s2s from Conception to RedactionDoc
        ws f2f from Conception to RedactionDoc
        ws f2s from Conception to Development
        ws s2s from Conception to RedactionTest
        ws f2f from Development to RedactionTest
        rd concepteur number 3
        rd redacteur number 1
        rd developpeur number 2
        rd testeur number 2
        rd machine number 4
        ri type concepteur activity Conception instances 2
        ri type machine activity Conception instances 2
        ri type redacteur activity RedactionDoc instances 1
        ri type machine activity RedactionDoc instances 1
        ri type developpeur activity Development instances 2
        ri type machine activity Development instances 3
        ri type testeur activity RedactionTest instances 1
        ri type machine activity RedactionTest instances 2
}
\end{verbatim}

A partir du modèle décrit textuellement, on peut l'exporter vers un format générique tel que XMI :

\begin{figure}[!h] 
\begin{center}
\includegraphics[width=10cm]{Capture-12.png}
\caption{Exemple de modèle SimplePDL au format XMI sous forme arborescente} 
\label{img1} 
\end{center}
\end{figure} 

\newpage

\section{T4 : Contraintes OCL sur le méta-modèle SimplePDL}

On complète ici les contraintes OCL pour capturer les contraintes qui n’ont pu l’être par le méta-modèle.

\subsection{Nom des ressources}

Comme pour les activités, il s'agit de vérifier que les ressources ont un nom unique.
On effectue donc un produit carthésien entre chaque élément de type \textit{RessourceDefinition}.\\

\begin{verbatim}
-- Invariant vérifiant si toutes les ressources ont un nom unique
inv resourcesNamesAreDifferent:
  let elements : Set(ProcessElement) = self.processElements in
  let rds: Set(ProcessElement) = elements->select(e | e.oclIsTypeOf(RessourceDefinition)) in
    rds->forAll(rd1, rd2: ProcessElement |
      if rd1 <> rd2 then
        rd1.oclAsType(RessourceDefinition).name <> rd2.oclAsType(RessourceDefinition).name
      else
        true
      endif
    )
\end{verbatim}

\subsection{Verification de la consistance}

On peut compléter la requète OCL des TPs, pour s'assurer que tous les éléments de type \textit{WorkDefinition} associés à une \textit{RessourceInstance} et les activités pointées par une \textit{WorkInstance} appartiennent bien au même \textit{Process}.

\begin{verbatim}
inv containmentConsistency:
  let elements : Set(ProcessElement) = self.processElements in
  let wds: Set(ProcessElement) = elements->select( e | e.oclIsTypeOf(WorkDefinition)) in
  let wss: Set(ProcessElement) = elements->select( e | e.oclIsTypeOf(WorkSequence)) in
  let ris: Set(ProcessElement) = elements->select( e | e.oclIsTypeOf(RessourceInstance)) in
    -- WS linked to a process WD are elements of this process
    wds->forAll(wd: ProcessElement |
      elements->includesAll(wd.oclAsType(WorkDefinition).linksToPredecessors)
      and elements->includesAll(wd.oclAsType(WorkDefinition).linksToSuccessors)
      and elements->includesAll(wd.oclAsType(WorkDefinition).linksToRessources))
    -- source and target of a process WS are elements of this process
    and wss->forAll(ws: ProcessElement |
       elements->includes(ws.oclAsType(WorkSequence).predecessor)
       and elements->includes(ws.oclAsType(WorkSequence).successor))
    -- activity and type of a process RI are elements of this process
    and ris->forAll(ws: ProcessElement |
       elements->includes(ws.oclAsType(RessourceInstance).type)
       and elements->includes(ws.oclAsType(RessourceInstance).activity))
\end{verbatim}

\subsection{Consistance des ressources}

On doit s'assurer qu'un élément de type \textit{WorkDefinition} ne demande pas plus de ressources que le nombre maximal possible.
Par exemple, si on dispose de 4 machines au total et que la \textit{WorkDefinition} conception en demande 6, le modèle n'est pas correct.

\begin{verbatim}
context RessourceInstance
inv ressourceCheck:
        if self.instances <= self.type.number then
                true
        else
                false
        endif
\end{verbatim}

\subsection{Vérification temporelle}

On doit s'assurer que la durée minimale d'une \textit{Process} ou d'une \textit{WorkDefinition} est bien inférieure à la durée maximale.

\begin{verbatim}
context WorkDefinition
inv timeCheck1:
        self.max_time < self.min_time implies
                false
\end{verbatim}

\section{T5 : Transformation SimplePDL vers PetriNet}

Il faut à présent engendrer une nouvelle transformation SimplePdl vers Petrinet. En effet, on doit reporter les changements réalisés dans méta-modèle SimplePdl, vers le méta-modèle PetriNet.\\

Nous allons de nouveau distinguer le cas des ressources de celui du temps.

\subsection{Ressources}

Pour les ressources, nous décidons de mettre en place le modèle suivant. Considérons deux types de ressources (\textit{RessourceDefinition}), machine et testeur.
En admettant que une \textit{WorkDefintion} (Conception dans cette exemple) nécéssite deux machines et un testeur, on aura le modèle suivant en PetriNet :\\

\begin{figure}[!h] 
\begin{center}
\includegraphics[width=10cm]{petri1.png}
\caption{Gestion des ressources} 
\label{img1} 
\end{center}
\end{figure} 

Basiquement on aura donc une règle permettant de parcourir les éléments de type \textit{RessourceDefinition}, qui seront traduites en éléments de type \textit{Place} au sens des réseaux de Petri. L'attribut \textit{nombre} de \textit{RessourceDefinition} correspond directement au marquage initial de la place créée.

\begin{verbatim}
rule RessourceDefinition2PetriNet {
        from rd: simplepdl!RessourceDefinition
        to
                p_ressource: petrinet!Place(
                        name <- rd.name,
                        marking <- rd.number,
                        net <- rd.getProcess())
}
\end{verbatim}

Il nous faut ensuite une règle qui pour chaque WorkInstance va créer un arc entre la place de la ressource et la transition de début de la WorkDefinition associée.
Il faut aussi crée un élément de type \textit{Arc} entre la transition de fin de la \textit{WorkDefinition} et la place de la ressource. Cet arc traduit le fait que les ressources sont libérées à la fin de l'exécution de l'activité. Il est évident qu'à la libération, on doit rendre autant de jetons que ceux qui ont été prélevés au démarrage de l'activité.

\begin{verbatim}
rule RessourceInstance2PetriNet {
        from ri: simplepdl!RessourceInstance
        to
                t_ressource_start: petrinet!Arc(
                        source <- thisModule.resolveTemp(ri.type, 'p_ressource'),
                        target <- thisModule.resolveTemp(ri.activity, 't_start'),
                        weight <- ri.instances,
                        kind <- #normal,
                        net <- ri.getProcess()),
                t_ressource_finish: petrinet!Arc(
                        source <- thisModule.resolveTemp(ri.activity, 't_finish'),
                        target <- thisModule.resolveTemp(ri.type, 'p_ressource'),
                        weight <- ri.instances,
                        kind <- #normal,
                        net <- ri.getProcess())
}
\end{verbatim}

\subsection{Temps}

Pour modéliser la gestion du temps dans un réseau de Petri, nous allons devoir mettre en place un observateur. Le principe de l'observateur est qu'il va nous fournir des informations sur le système sans pour autant en altérer le comportement. Typiquement, l'observateur d'une activité va nous informer sur le fait qu'elle se finit avec une durée correcte (comprise dans l'intervalle [\textit{temps\_min},\textit{temps\_max]}), trop courte (durée exécution inférieure à la durée minimale) ou trop longue (durée exécution inférieure à la durée maximale).

Prenons l'activité \textit{Conception} à laquelle on adjoint un observateur :

\begin{figure}[!h] 
\begin{center}
\includegraphics[width=10cm]{Capture-18.png}
\caption{Observateur sur l'activité Conception} 
\label{img1} 
\end{center}
\end{figure} 

Les observateurs sont donc présents au niveau de chaque activité puisqu'on veut surveiller pour chacune leur temps d'exécution, mais aussi pour le processus tout entier qui possède également des attributs \textit{time\_min} et \textit{time\_max}. Pour observer le processus tout entier, le principe est le même que pour une activité, car à l'instar d'une activité, il possède les places \textit{Ready}, \textit{Started} et \textit{Terminated} (équivalent de \textit{Finished} pour les activités), \\

Dans notre exemple, on obtient le réseau de Petri suivant modélisant la gestion des ressources et l'observation des temps d'exécution :

\begin{figure}[!h] 
\begin{center}
\includegraphics[width=18cm]{Capture-17.png}
\caption{Réseau de Petri de l'exemple de processus sous l'outil Tina} 
\label{img1} 
\end{center}
\end{figure} 

\newpage

Le code à fournir pour mettre en place cet observateur est assez long. Il n'est donc pas cité, mais est présent dans les rendus de ce BE.

\section{T6 : Validation de la transformation SimplePDL vers PetriNet}

L'étape de validation consiste à appliquer la transformation précédente et à l'appliquer à un exemple. Prenons le fichier de modèle suivant qui reprend l'exemple du sujet.

\begin{figure}[!h] 
\begin{center}
\includegraphics[width=10cm]{Capture-7.png}
\caption{Modèle SimplePDL conforme à l'exemple du sujet} 
\label{img1} 
\end{center}
\end{figure} 

Nous obtenons un fichier .PetriNet correspondant bien à ce qui était attendu. Il est valide par rapport au méta-modèle PetriNet.
On peut donc valider cette transformation et attaquer la partie PetriNet2Tina.

\section{T7 : Transformation PetriNet vers Tina}

Le méta-modèle PetriNet n'ayant pas évolué, il n'y a pas non plus de raisons de faire évoluer la transformation PetriNet vers Tina. 
Nous avons juste du apporter quelques retouches par rapport à la version des TP. En effet, nous avions considéré que le temps associé par défaut à une transition est par défaut \textit{[0,0]} sous Tina. Pourtant, il s'agit d'un cas particulier puisque nous voulons que par défaut nos transitions soient de type \textit{[0,w[}.\\

Dans cette transformation, on doit donc s'assurer que dans le modèle PetriNet si \textit{temps\_max = -1} (signifie \textit{w} dans la syntaxe Tina car \textit{temps\_max} forcément entier), alors le fichier de sortie comportera un intervalle terminé par \textit{w[}.\\

\begin{verbatim}
-- translate a transition to Tina syntax.
helper
context petrinet!Transition
def: toTina(): String =
  let inNodesNames: Sequence(String)
    = self.incomings->collect(arc | arc.asTina(true)) in
  let outNodesNames: Sequence(String)
    = self.outgoings->collect(arc | arc.asTina(false) in
  ('tr '  + self.name
  + ' [' + self.min_time + ',' + self.max_time + ']'
  + thisModule.concatenateStrings(inNodesNames, ' ', '')
  + ' -> '
  + thisModule.concatenateStrings(outNodesNames, ' ', '')).regexReplaceAll('-1]', 'w[')
;
		
-- Translate an Arc to Tina syntax. isSource indicates if it is an
-- outgoing arc or an incoming one
helper
context petrinet!Arc
def: asTina(isSource: Boolean): String =
  if isSource = true then
    self.source.name
  else
    self.target.name
  endif
  +
  if self.kind = #normal then
    '*'
  else
    '?'
  endif
  + self.weight.toString()
;
\end{verbatim}

\section{T8 : Contraintes OCL sur le méta-modèle PetriNet}
Il s'agit ici de vérifier certaines propriétés fondammentales que l'on ne peut exprimer dans le méta-modèle PetriNet, c'est pourquoi l'on s'appuie sur des règles OCL qui permettent de rajouter certaines contraintes et de vérifier si les modèles instances du méta-modèle PetriNet les respectent.\\

\subsection{Noms du réseau, des places et transitions}
Comme pour le méta-modèle SimplePDL, on vérifie que les noms sont du réseau de Petri, des éléments de type \textit{Place} et de type \textit{Transitions} ne sont pas vides. On vérifie également que les noms des éléments de type \textit{Place} sont uniques, de même pour les éléments de type \textit{Transition}.

\subsection{Liaison des arcs}
Il faut vérifier qu'un élément de type \textit{Arc} lie un élément de type \textit{Place} avec un élément de type \textit{Transition}, dans un sens ou dans l'autre. Il faut donc vérifier pour chaque élément de type \textit{Arc} que l'on vérifie l'une des deux conditions suivantes:
\begin{itemize}
\item l'attribut \textit{source} est un élément de type \textit{Place} et l'attribut \textit{target} est un élément de type \textit{Transition}
\item l'attribut \textit{source} est un élément de type \textit{Transition} et l'attribut \textit{target} est un élément de type \textit{Place}
\end{itemize}
En langage OCL, cette propriété se vérifie avec la règle suivante :

\begin{verbatim}
context Arc
inv arcValid:
-- la source d'un read_arc est toujours de type place et la destination de type transition
        (self.kind = ArcKind::read_arc implies 
                (self.source.oclIsTypeOf(Place) and self.target.oclIsTypeOf(Transition)))
        and 
-- la source et la destination d'un arc normal doivent etre de nature opposées 
        (self.kind = ArcKind::normal implies 
                (self.source.oclIsTypeOf(Place) and self.target.oclIsTypeOf(Transition)
                or
                self.source.oclIsTypeOf(Transition) and self.target.oclIsTypeOf(Place)))
\end{verbatim}

\subsection{Composantes temporelles des transitions}

Il faut également vérifier que si elles sont toutes deux entiers positifs, la borne temporelle inférieure d'une transition doit être inférieure ou égale à la borne supérieure. On peut traduire la propriété en OCL de la manière suivante :

\begin{verbatim}
\end{verbatim}

\section{T9 : Propriétés LTL}

Maintenant que nous avons effectué des vérification sur les modèles SimplePDL et PetriNet, il faut maintenant effectuer une vérification des propriétés de sûreté sur le modèle de réseau de Petri servant d'entrée à l'application Tina. Ces vérification peuvent s'effectuer via des requêtes LTL exécutées sur l'outil Selt.

\subsection{Relation entre terminaison des activités et mort du réseau}

Pour vérifier la terminaison d'un processus, il s'agit d'exécuter plusieurs requètes LTL.
Soit l'opérateur finished qui correspond au ET logique entre toutes les places de fin de chaque activité correspondant à leur état \textit{Finished}, on a les propriétés suivantes :\\

\begin{verbatim}
# Si l'on a fini, alors le réseau est mort, doit renvoyer True.
[] (finished => dead);

# Il est possible que le réseau soit mort, doit renvoyer True.
[] <> dead;

# Si le réseau est mort, alors toutes les activités sont terminées, doit renvoyer True.
[] (dead => finished);

# On atteindra jamais l'état où toutes les activités sont terminées, doit renvoyer False.
- <> finished;
\end{verbatim}

\subsection{Invariance des activités}

On souhaite également pouvoir vérifier qu'une fois le processus démarré (\textit{Started}), ses activités seront toujours soit dans l'état :
\begin{itemize}
\item non commencée - \textit{Ready}
\item en cours - \textit{Started}
\item terminée - \textit{Finished}
\end{itemize}
Dans notre exemple, cette propriété se traduit par les requêtes LTL suivantes :

\begin{verbatim}
[] (exemple_started => (Conception_ready \/ Conception_started \/ Conception_finished));
[] (exemple_started => (RedactionDoc_ready \/ RedactionDoc_started \/ RedactionDoc_finished));
[] (exemple_started => (Developpement_ready \/ Developpement_started \/ Developpement_finished));
[] (exemple_started => (RedactionTests_ready \/ RedactionTests_started \/ RedactionTests_finished));
\end{verbatim}

\subsection{Irréversibilité des activités terminées}

Enfin, on peut vérifier qu'une activité terminée n'évolue plus du tout : c'est-à-dire qu'une activité ayant atteint l'état \textit{Finished} n'atteindra plus jamais les états \textit{Ready} ou \textit{Started}. Dans notre exemple, cela se traduit par les requêtes suivantes :

\begin{verbatim}
[] (Conception_finished => - <> (Conception_ready \/ Conception_started));
[] (RedactionDoc_finished => - <> (RedactionDoc_ready \/ RedactionDoc_started));
[] (Developpement_finished => - <> (Developpement_ready \/ Developpement_started));
[] (RedactionTests_finished => - <> (RedactionTests_ready \/ RedactionTests_started));
\end{verbatim}

\chapter{Partie 2 : Extensions supplémentaires}

Nous avons implémenté deux extensions supplémentaires qui apportent des fonctionnalités supplémentaires mais aussi de la compléxité au système. C'est pourquoi nous avons les avons implémenté en parallèle et façon à ce qu'elles soient indépendantes entre elles.

\section{Gestion plus fine des ressources}

\subsection{Transformation SimplePDL vers PetriNet}

\section{Ressources alternatives}

Il s'agit ici de mettre en oeuvre une gestion alternative des ressources. Le principe est qu'il existe plusieurs configurations possibles de ressources.
Ainsi, soit une configuration A nécéssitant 2 machines et 3 développeurs et une configuration B avec 3 machines et 2 testeurs, on pourra exprimer le fait qu'une WorkDefition a besoin que la configuration A ou que la configuration B soit effective.

\subsection{Redéfinition du méta-modèle}

Il est donc clair que l'on doit modifier le méta-modèle pour expliciter cette possibilité de ``regroupement des ressources''.

\begin{figure}[!h] 
\begin{center}
\includegraphics[width=15cm]{Capture-13.png}
\caption{Méta-modèle avec ressources alternatives} 
\label{img1} 
\end{center}
\end{figure} 

Et sous forme graphique,\\

\newpage

\begin{figure}[!h] 
\begin{center}
\includegraphics[width=8cm]{Capture-14.png}
\caption{Méta-modèle sous forme graphique} 
\label{img1} 
\end{center}
\end{figure} 

Pour permettre ce regroupement, on rajoute une classe RessourceConfig. Cette classe va contenir des liens vers des RessourceInstance. Ainsi, on va exprimer le fait qu'il peut exister plusieurs configurations mettant en oeuvre différents jeux de ressources.\\

On remplace donc le lien de WorkDefinition a RessourceInstance, par un lien de WorkDefinition a RessourceConfig. Ce lien a une multiplicité 0..\* car on peut avoir plusieurs Configurations pour une même WorkDefinition.

\subsection{Transformation SimplePDL vers PetriNet}

Prenons l'exemple suivant, 

\begin{figure}[!h] 
\begin{center}
\includegraphics[width=8cm]{Capture-15.png}
\caption{Exemple mis en oeuvre} 
\label{img1} 
\end{center}
\end{figure}

On dispose de 4 ressources : Machine, Analyste, Testeur, GenerateurTest.
Par souci de lisibilité, on limite le nombre de WorkDefinition, à une seule : redactionTest. On définit également deux configurations C1 et C2. C1 met en oeuvre deux WorkInstance (Machine, 2) et (GenerateurTest,1). C2 met en oeuvre trois WorkInstance (Analyste,1), (GenerateurTest,1) et (Machine,1).\\

Cet exemple va nous permettre d'illustrer les transformations qui doivent être effectuées.\\

Il faut donc reprendre la transformation SimplePDL vers PetriNet et transformer la gestion des Ressources. Toujours par souci de lisibilité, on s'affranchit de la génération d'observateurs.\\

Le principe du réseau de pétri gérant la configuration des ressources est le suivant :\\

\begin{itemize}
\item Il est possible de choisir entre chacune des configurations (si du moins il y a suffisament de ressources)
\item Une fois que l'on a choisi de démarrer avec une configuratio, on ne pourra plus choisir l'autre
\item Les Ressources utilisées doivent être restituées à la fin de l'activité, et qui plus est dans la bonne configuration.
\end{itemize}

Il ressort de cette analyse que l'on va devoir utiliser des places nous indiquant :\\

\begin{itemize}
\item Quelle configuration est choisie : ConfigurationName\_used
\item Une configuration a été choisie : ProcessName\_ress\_l\\
\end{itemize}

Et des transitions pour :\\

\begin{itemize}
\item Choisir une configuration : ConfigurationName.start
\item Rendre les ressources : ConfigurationName.finished\\
\end{itemize}

Pour l'exemple cité précédement, on obtient donc le réseau de pétri suivant :\\

\begin{figure}[!h] 
\begin{center}
\includegraphics[width=15cm]{Capture-16.png}
\caption{Réseau de pétri obtenu} 
\label{img1} 
\end{center}
\end{figure}

Bien entendu, le code de la transformation a été adapté pour fornir ces changements. Il est disponible en annexe.\\

Pour ce qui est de la transformation PetriNet2Tina, il n'y a toujours aucune raison de la modifier car, on a pu conserver la même structure pour le méta-modèle PetriNet.ecore.\\








\end{document}
